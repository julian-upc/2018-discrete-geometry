\documentclass[11pt]{amsart}

\usepackage{a4wide}
\usepackage{paralist}
\usepackage{url}
\usepackage{bbm}
\usepackage{nopageno}

\newcommand{\cA}{\mathcal{A}}
\newcommand{\cS}{\mathcal{S}}
\DeclareMathOperator{\conv}{conv}
\DeclareMathOperator{\New}{New}
\DeclareMathOperator{\area}{area}
\newcommand{\RR}{\mathbbm{R}}
\newcommand{\CC}{\mathbbm{C}}

\begin{document}
\begin{center}
\textbf{\sffamily
   Discrete and Algorithmic Geometry }

\medskip
   Julian Pfeifle,
   UPC, 2018
\end{center}


\begin{center}
  \textbf{\sffamily Sheet 2}

\bigskip
 due on Monday, November 19, 2018

\end{center}

\bigskip
\bigskip
\bigskip

\section*{Writing (due November 19)}
\begin{enumerate}
\item
  \begin{enumerate}[(a)]
  \item Using Gale's Evenness Criterion, explicitly write down all
    subsets of vertices that make up a facet of the $4$-dimensional
    cyclic polytope $C_4(8)$.
  \item Draw the dual graph of $C_4(8)$, i.e.\ the graph that has the facets of $C_4(8)$ as nodes and an edge between two nodes if the corresponding facets share a ridge.
    Try to make your drawing as legible as possible. \emph{Hint:} Labeling the nodes with the corresponding vertex sets will help you to get organized.
\end{enumerate}
  \bigskip\bigskip
\item Let $P$ be the \emph{24-cell}, $P = \conv\{\pm e_i\pm e_j : 1\le i\ne j\le 4\}$, where $(e_1,\dots,e_4)$ is the standard basis of~$\RR^4$.
 Also, consider the \emph{simple roots of type $F_4$}, namely the rows $r_1,\dots,r_4$ of the matrix
    \[
      \begin{bmatrix}
        1 & -1 & 0 & 0 \\
        0 & 1 & -1 & 0 \\
        0 & 0 & 1 & 0 \\
        -\frac12 & -\frac12 & -\frac12 & -\frac12
      \end{bmatrix},
    \]
    and the linear hyperplanes $H_1,\dots,H_4$ orthogonal to them.
    \begin{enumerate}
      \item Count the vertices of $P$.
      \item The \emph{root system of type $F_4$} arises from the four
        simple roots by reflecting them in the $H_i$, adding all
        new vectors (and the linear hyperplanes orthogonal to them)
        obtained in this way to the set of reflecting hyperplanes, and
        repeating the process until no new normal vectors/hyperplanes
        are found. Count the number of resulting
        hyperplanes. \emph{Answer: $48$}
      \item Show that reflecting the vertices of $P$ in the $H_i$ leaves $P$ invariant.
      \item Find all vertices of $P$ inside the \emph{fundamental cone of~$F_4$},
        \[
          C
          \ = \
          \{x\in\RR^4:\langle r_i,x\rangle\ge0\text{ for }i=1,\dots,4\}.
          \]
      \item Use this to efficiently describe and count the number of facets of~$P$.
      \end{enumerate}
\end{enumerate}
\end{document}
 