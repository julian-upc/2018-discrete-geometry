\documentclass[10pt,a4paper]{article}
\usepackage[utf8]{inputenc}
\usepackage{amsmath}
\usepackage{amsfonts}
\usepackage{amssymb}
\usepackage{bbm}
\usepackage{dsfont}
\usepackage{amsthm}  %Esto sirve para utilizar simbolos
\usepackage{graphicx}
\usepackage{float}
\usepackage{color}
%\usepackage{subfigure}
\usepackage[left=2cm,right=2cm,top=2cm,bottom=2cm]{geometry}
%\usepackage{caption}
\usepackage{hyperref}
\usepackage{cleveref}
\usepackage{todonotes}
%\usepackage{cancel}
\usepackage[english]{babel}
\usepackage{float}
\usepackage{mathtools}
%\usepackage{subfigure} 
\usepackage{subcaption}
\usepackage{textcomp}
\usepackage{bm}
\usepackage[nobreak=true]{mdframed}	
%\usepackage{minted}
%\usemintedstyle{mathematica}

\usepackage{fancyhdr}
	\pagestyle{fancy}

\hypersetup{
	colorlinks=false,
	pdfborder={1 1 0.005},
}
	
\newcommand{\cA}{\mathcal{A}}
\newcommand{\cS}{\mathcal{S}}
\DeclareMathOperator{\New}{New}
\DeclareMathOperator{\area}{area}
\DeclareMathOperator{\vol}{vol}
\DeclareMathOperator*{\argmax}{arg\,max}
\def\defs{\stackrel{\tiny{\mbox{def}}}{=}}		% For definitions
	
\DeclareMathOperator{\conv}{conv}
%\renewcommand{\theenumi}{(\alph{enumi})}
\newcommand{\RR}{\mathbb{R}}
\theoremstyle{plain}
\newmdtheoremenv[linewidth = 1pt]{result}{Result}
\newtheorem*{theorem*}{Theorem}
\newtheorem{theorem}{Theorem}
\newtheorem*{prop}{Proposition}
\newtheorem{lemma}{Lemma}
\newtheorem{corollary}{Corollary}
\theoremstyle{remark}
\newtheorem{fact}{Fact}
\newtheorem{claim}{Claim}
\newtheorem{remark}{Remark}

\theoremstyle{definition}
\newtheorem{definition}{Definition}


\begin{document}
\thispagestyle{plain}
\begin{center}
\rule{\linewidth}{0.05mm}\
{\Large \textbf{Discrete and Algorithmic Geometry: Sheet 4\\}}
{\large Ander Elkoroaristizabal Peleteiro \& Filip Cano Córdoba \& Alberto Larrauri Borroto\\}
\rule{\linewidth}{0.05mm}\
\end{center}

\begin{enumerate}
	\item Definition 9.2 in Ziegler's \emph{Lectures on Polytopes} constructs the linear map
	\[
	P
	\ \xrightarrow{\pi^c}\ 
	Q^c :=
	\Big\{ \binom{\pi(x)}{cx} : x\in P\Big\}
	\ \subset \
	\RR^{q+1}
	\]
	from a projection $\pi:P\subset\RR^p\to Q\subset\RR^q$ and a linear function $c\in(\RR^p)^\star$.
	Is it possible to give an algorithm to determine the set of lower faces 
	$\mathcal L^\downarrow(Q^c)$ of $Q^c$ 
	from just the set of facet normals of~$Q$, 
	the projection~$\pi$, and the linear function~$c$, without running a convex hull algorithm on $Q^c$?
	
	\bigskip\bigskip
	\item Show that
	\[
	\int_P f(x)\,\text{d}x
	\ = \
	\vol(P) \cdot f(p_0)
	\]
	for any polytope $P$ and linear function $f$, 
	where $p_0 = \frac{1}{\vol(P)}\int_P x\,\text{d}x$ denotes the barycenter of $P$.
	\bigskip\bigskip
	\item Complete the proof of Theorem 9.6 in Ziegler's \emph{Lectures on Polytopes}, 
	possibly referring to \cite{bs-1992}.
	
\end{enumerate}


\textbf{1.}

It is not possible to give such an algorithm.

This is because $\pi,c$ and the set of facets of $Q$ do not determine the lower faces of $Q^c$.
Consider $\pi:\RR^2\to \RR$ that deletes the last coordinate. 
Consdier then $c=(0,1)$. 
In this case, $\pi^c$ is the identity in $\RR^2$.
Since in this case $q=1$, the interval is the only polytope the set of facet normals of $q$ is always the same,
so $q=1$, the only relevant information is $\pi$ and $c$, but in this case, $\pi^c$ is the identity.
Therefore, it such algorithm existed, the set of lower faces would be the same for all polygons,
which is not true.

\textbf{2.}

Using the fact that $f$ is linear and linearity of the integral:
\begin{equation}
	\vol(P) f(p_0)
	= \vol(P) f\left( \frac1{\vol (P)}\int_P x\,\mathrm{d}x \right)
	= f\left(\int_P x\,\mathrm{d}x \right) 
	= \int_P f(x) \,\mathrm{d}x
\end{equation}

\textbf{3. }

IN THE NEXT EPISODE.

\bibliographystyle{amsplain}
\bibliography{bib}

\end{document}



