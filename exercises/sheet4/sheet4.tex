\documentclass[11pt]{article}
\usepackage[utf8]{inputenc}
\usepackage{ dsfont }
\usepackage{ amssymb }
\usepackage{ tipa }
\usepackage[utf8]{inputenc}
\usepackage{ textcomp }
\usepackage{ stmaryrd }
\usepackage{graphicx}
\usepackage{wrapfig}
\graphicspath{{/home/} }
\usepackage{amssymb,amsmath,amsthm}
\usepackage{mdframed}
\usepackage{systeme,mathtools}
\usepackage{multicol}
\usepackage{ dsfont }
\usepackage{enumerate}
\usepackage{booktabs}
\newcommand\T{\textrm{T}}  % "true"
\newcommand\F{\textrm{F}}  % "false"
\usepackage{multirow}
\usepackage{hyperref}
\usepackage{tikz-cd}
\usepackage{caption}
\usepackage{titlesec}
\usepackage{subcaption}
\usepackage{ upgreek }
\usepackage{mathtools}
\usepackage{amsmath}
\usepackage[all]{xy}
\usepackage{pst-node}
\usepackage{tikz-cd} 
\usepackage{stackrel}
\usepackage[margin=1.0in]{geometry}
\usepackage[english]{babel}
\usepackage{amsthm}
\usepackage{amsmath}
\usepackage{eufrak}
\usepackage{afterpage}
\usepackage{a4wide}
\usepackage{paralist}
\usepackage{url}
\usepackage{bbm}
\usepackage{nopageno}


\providecommand{\abs}[1]{\lvert#1\rvert}
\newcommand\acc[2]{\ensuremath{{}^{#1}\hskip-0.1ex{#2}}}
\renewcommand{\baselinestretch}{1.2}
\newtheorem{theorem}{Theorem}[section]
\newtheorem{remark}{Remark}[section]
\newtheorem{corollary}{Corollary}[section]
\newtheorem{lemma}{Lemma}[section]
\newtheorem{definition}{Definition}[section]
\newtheorem{proposition}{Proposition}[section]
\newtheorem{example}{Example}[section]


\newcommand{\cA}{\mathcal{A}}
\newcommand{\cS}{\mathcal{S}}
\DeclareMathOperator{\conv}{conv}
\DeclareMathOperator{\New}{New}
\DeclareMathOperator{\area}{area}
\DeclareMathOperator{\vol}{vol}
\newcommand{\RR}{\mathbbm{R}}
\newcommand{\CC}{\mathbbm{C}}
\DeclareMathOperator{\ext}{ext}
\DeclareMathOperator{\Ext}{Ext}
\DeclareMathOperator{\Hom}{Hom}
\DeclareMathOperator{\Aut}{Aut}
\DeclareMathOperator{\pear}{Par}
\DeclareMathOperator{\Pear}{par}
\DeclareMathOperator{\sign}{sgn}
\DeclareMathOperator{\im}{im}
\DeclareMathOperator{\id}{id}




\title{DAG \\Sheet 4}
\author{Víctor Ruiz}
\date{}

\usepackage{natbib}
\usepackage{graphicx}

\begin{document}

\maketitle

\begin{enumerate}
\item Definition 9.2 in Ziegler's \emph{Lectures on Polytopes} constructs the linear map
  \[
    P
    \ \xrightarrow{\pi^c}\ 
    Q^c :=
    \Big\{ \binom{\pi(x)}{cx} : x\in P\Big\}
    \ \subset \
    \RR^{q+1}
  \]
  from a projection $\pi:P\subset\RR^p\to Q\subset\RR^q$ and a linear function $c\in(\RR^q)^\star$.
  Is it possible to give an algorithm to determine the set of lower faces $\mathcal L^\downarrow(Q^c)$ of $Q^c$ from just the set of facet normals of~$Q$, the projection~$\pi$, and the linear function~$c$, without running a convex hull algorithm on $Q^c$?

  \bigskip\bigskip
\item Show that
  \[
    \int_P f(x)\,\text{d}x
    \ = \
    \vol(P) \cdot f(p_0)
  \]
  for any polytope $P$, where $p_0 = \frac{1}{\vol(P)}\int_P x\,\text{d}x$ denotes the barycenter of $P$.
  \bigskip\bigskip
\item Complete the proof of Theorem 9.6 in Ziegler's \emph{Lectures on Polytopes}.
  
\end{enumerate}




\begin{enumerate}

\item It is not possible, since we cannot obtain $Q^c$ (or $\mathcal L^\downarrow(Q^c)$) just from $Q$, $\pi$ and $c$. \newline For example, take $q = 1$, $Q$ any 1-polytope (any interval), $\pi(x,y) = x$ and $c = (0,1)$, then $\pi^c$ is the identity in $P$, so $Q^c = P$, but we are not given $P$, so it's impossible to get $Q^c$ or just $\mathcal L^\downarrow(Q^c)$.

\item Directly from the linearity of $f$ and the linearity property of the integral:

\begin{equation*}
	\vol(P) f(p_0)
	= \vol(P) f\left( \frac1{\vol (P)}\int_P x\,\mathrm{d}x \right)
	= f\left(\int_P x\,\mathrm{d}x \right) 
	= \int_P f(x) \,\mathrm{d}x
\end{equation*}

\item I will enumerate the parts of the sketch of Ziegler's proof which are not clear and I'll explain them:

\begin{enumerate}[(i)]

\item $\Sigma(P,Q)$ is convex.

\begin{proof}
Given two points $a,b \in \Sigma(P,Q)$ and their corresponding sections $\gamma_a, \gamma_b$. Then $(ta + (1-t)b) = \int t\gamma_a + (1-t)\gamma_b = \int \gamma dx$ since the convex combination of 2 sections is a section (by the linearity of the integral and $\pi$).
\end{proof}

\item $\dim(\Sigma(P,Q)) \leq p-q$ with $\dim(P) = p$, $\dim(Q) = q$.

\begin{proof}
We have to see that $\pi^{(-1)}(r_0)$ has dimension p-q. Observing that, since we have supposed that the dimensions of $P$ and $Q$ are maximal, $\ker(\pi) = p - q$ and so we get the result.
\end{proof}



\item "Every piecewise linear section that is not tight can be changed locally in two directions; thus it can be written as a convex combination of two other sections".

\begin{proof}

\end{proof}


\item "To detect the vertices of $\Sigma(P, Q)$, we use that they arise as the unique maxima for generic linear functions $c \in (\mathbb{R}^p)^*$. However, if $c$ is generic, then
every fiber $\pi^{-1}(r)$ for $ r \in Q$ has a unique maximal element with respect to $c$".

\begin{proof}
The problem of this part is the definition of $generic$ linear functions. In this case there's only one possibility for the definition of generic:

\begin{definition}
A linear function $c$ will be generic with respect to $\Gamma(P,Q)$ if $c$ is generic with respect to $P$ and with respect to $\pi$, that is that $c$ is not parallel to $\pi^{(-1)}(r)$ for $r \in Q$.
\end{definition}

Now with this definition of generic, it is clear that every fiber $\pi^{-1}(r)$ has a unique maximal element with respect to $c$. It is "generic" since the set of forbidden $c$ has area 0.
\end{proof}


\item "A simple computation shows that for a continuous section $\gamma:Q \rightarrow P$, the point $\frac{1}{\vol(Q)}\int \gamma dx \in \Sigma(P,Q)$ lies in the face defined by $c$ if and only if the image of the section is entirely contained in the collection of faces $F^c \subseteq L(P)$".

\begin{proof}

\end{proof}


\end{enumerate}


\end{enumerate}




\end{document}

