\documentclass[12pt,a4paper]{article}

\usepackage{enumerate}
\usepackage{graphicx}
\usepackage{float}
\usepackage[a4paper, margin=3.5cm]{geometry}
\setlength{\parindent}{2em} 
\setlength{\parskip}{1em} 
\usepackage{amsfonts}
\usepackage{amssymb}
\usepackage{amsmath}
\usepackage{comment}

\newcommand{\cA}{\mathcal{A}}
\newcommand{\cS}{\mathcal{S}}
\newcommand{\RR}{\mathbb{R}}
\newcommand{\CC}{\mathbbm{C}}
\newcommand{\vol}{\text{vol}}
\newcommand{\id}{\text{id}}

\begin{document}

\noindent
\textbf{UPC MAMME / Discrete Geometry 2018}

\noindent
\textbf{Sheet 4}

\noindent
\textbf{Diego Acedo and Xavier Gombau}


\subsubsection*{Solution 1}

There is one thing which we do not find entirely clear in this problem. 
The question whether we are given $P$ or not.

If $P$ is not given, we have discussed with other colleagues the problem, and we have come to the conclusion that the lower faces of $Q^c$ are not determined from only the linear map $\pi : \RR^p \longrightarrow \RR^{q+1} $, and so it is not possible to give an algorithm to find such faces. 

% Posibilidad de explicarlo mejor

Nevertheless, we think that since $\pi$ is defined on $P$, then $P$ is given along with $\pi$, and so the lower faces are determined. 
We have found a (computationally bad) algorithm which determines the set of lower faces of $Q^c$, but, since we cannot distinguish the interior points of $Q^c$ from those in the boundary, we use at some point a convex hull algorithm to solve this. 
We are working on a way to solve that problem or show that indeed you are forced to use such convex hull algorithm. 

Recall that the lower faces of $Q^c$ are those whose valid inequality is given by a hyperplane $h$ with $h_{d+1} < 0$ and a vector $h_0$, with $x \cdot h \leq h_0$ for all $x \in Q^c$.

We do the following: 

\begin{enumerate}
	\item For each vertex $x_i \in P$, compute $\pi^c(x_i)$, so then it is $Q^c = conv\{\pi^c(x_i)\}$.
	
	\item Use a convex hull algorithm to find the actual vertices of $Q^c$.
	
	\item For all subsets of $d+1$ vertices:
	\begin{enumerate}
		\item Check if they are in the same hyperplane.
		if they are, compute the hyperplane $(h,h_0)$ with $h_{d+1} < 0$
		
		\item For all other points $v$, check if $v \cdot h \leq h_0$:
		\begin{enumerate}
			\item If not, then it cannot be a lower facet, so we stop.
			\item If $v \cdot h = h_0$, then $v$ is in the hyperplane, so we add $v$ to the subset, and compute the convex hull of this new subset to find the real facet.
		\end{enumerate}
		If at the end, all $v$ verify $v \cdot h \leq h_0$, then we have a valid inequality, and thus a face, and this face will be a lower face by definition.
				
	\end{enumerate}
		
\end{enumerate}



\subsubsection*{Solution 2}
Since $f$ is linear, we can write $f(x)=Ax$ for some matrix $A$. Then,
\[ f(x) = Ax = \left( \sum_{i=1}^{n} a_{1i}x_i, \dots, \sum_{i=1}^{n} a_{ni}x_i \right) = (f_1(x), \dots, f_n(x)). \]
For each coordinate $j=1,\dots,n$, using the linearity of the integral, 
\[ \int_{Q}^{} f_j(x) \ dx = \int_{Q} \sum_{i=1}^{n} a_{ji}x_i \ dx = \sum_{i=1}^{n} a_{ji} \int_{Q} x_i dx = \sum_{i=1}^{n} a_{ji} \cdot r_{0i} = \text{vol}(Q)  A_j \cdot r_0, \]
where $A_j$  is the $j$th row of $A$. Then, 
\[ \int_{Q}^{} f(x) \ dx = (A_1\cdot r_0 \text{vol}(Q), \dots, A_n\cdot r_0 \text{vol}(Q)) = A \cdot r_0 \cdot \text{vol}(Q) = f(r_0) \cdot \text{vol}(Q). \]
\subsubsection*{Solution 3}

We follow the proof formally proving Ziegler's claims. 

First of all, any convex combination of two sections is a section again.
If $\gamma_1, \gamma_2$ are both sections of $\pi$, then for any $t\in [0,1]$, since $\pi$ is linear,
$$\pi \circ (t\gamma_1+(1-t)\gamma_2)= t(\pi\circ \gamma_1) + (1-t)(\pi \circ \gamma_2) = t\cdot \text{id}+(1-t)\cdot\text{id}=\text{id}.$$

Now, by the linearity of the integral, $\Sigma(P,Q)$ is convex. 
If $p,q\in \Sigma(P,Q)$, then we can write
$$
\begin{array}{cc}
 p=\frac{1}{\text{vol}(Q)}\int_Q \gamma_p(x) dx, & q=\frac{1}{\text{vol}(Q)}\int_Q \gamma_q(x) dx
\end{array}
$$
and for any $t\in[0,1]$, 
$$tp+(1-t)q = \frac{1}{\text{vol}(Q)}\int_Q (t\gamma_p + (1-t)\gamma_q)(x) dx \in \Sigma(P,Q)$$
Now, the dimension of $\Sigma(P,Q)$ cannot be larger than $\dim P - \dim Q$, since $\Sigma(P,Q)\subset \pi^{-1}(r_0)$, which has this dimension. Here there are two assertions not proven. 
First of all, that $\Sigma(P,Q)\subset \pi^{-1}(r_0)$, and secondly that $\dim \pi^{-1}(r_0)=\dim P - \dim Q$. 
The second assertion is obtained by noticing that, by the first isomorphism theorem, $\pi$ induces an isomorphism $\RR^p/\ker \pi \simeq \RR^q$, so assuming $P$ and $Q$ are full dimensional, it must be $\dim P - \dim Q = \dim \ker \pi = \dim \pi^{-1}(r_0)$. 
For the first assertion, take $p\in \Sigma(P,Q)$, along with its associated section $\gamma$. 
Then, 
$$\pi(p)=\frac{1}{\vol Q} \int_Q (\pi \circ \gamma)(x) dx = \frac{1}{\vol Q} \int_Q x dx =r_0.$$

At this point, Ziegler claims: "Every piecewise linear \textit{non tight} section can be changed locally in two opposite directions; thus it can be written as a convex combination of other two sections that have a different integral. Thus we get that the set $\Sigma(P, Q)$ is the convex hull of the integrals $\frac{1}{\vol Q}\int_Q \gamma(x)dx$ for which $\gamma$ is a tight (piecewise linear, continuous) section. From this we conclude that $\Sigma(P, Q)$ is a polytope."

First thing to notice here is that Ziegler does not define what a tight section is. 
We have searched for possible definitions and we have several candidates: a measure thoretic idea, and another two trying to use the definition of tight subdivision. 
We will keep updating the document as we progress.

\begin{comment}
	Following the idea that a subdivision is tight if $\dim F = \dim \pi(F)$ for all $F \in \mathcal{F}(P)$, we define a linear section $\gamma$ to be tight if $\dim F = \dim \gamma(F)$ for all $F\in \mathcal{F}(Q)$, and so a piecewise linear section will be tight if it is in each piece. 
	Another option, using the measure theoretic definition of tightness, would be to say that a section $\gamma$ is tight if for any $\epsilon >0$ there exists a compact subset $K_\epsilon\subset Q$ such that the Lebesgue measure $\mu(\gamma(Q\setminus K_\epsilon))<\epsilon$. In any case, we are still working on understanding what these definitions actually mean and how any of them would correspond to both of Ziegler's claims.
\end{comment}

Now, to find the vertices of $\Sigma(P,Q)$ we use the fact that they correspond to a valid inequality where equality is reached by a single point, and so they maximize a certain linear form $c\in (\RR^p)^*$ when restricted to $P$. 
Then again, Ziegler uses the term \textit{generic} with a certain amount of flexibility. We think that in this case it is enough that the corresponding hyperplane associated to $c$ is not parallel to any of the faces in $P$ (i.e. it is in general position with respect all facet defining hyperplanes). 
If this is the case, then any linear form cannot be maximized simultaneously in two or more points, and so it will only be maximize in some vertex. 
Also, Ziegler claims that if $c$ is generic, then every fiber $\pi^{-1}(r)$ for $r \in Q$ has a unique maximal element with respect to $c$. 
We are still trying to prove this, but maybe one needs to have more flexibility with the word \textit{generic} in order to prove this.
If all of this is true, then one can define a section $\gamma^c(r)$ as the point maximizing $c$ in $\pi^{-1}(r)$ for each $r\in Q$. 
Here Ziegler claims that this section is unique (clear, buy construction), coherent and tight. 
Since again, we do not have a definition for coherent section, we are bound to believe that a section induces some subdivision, and we can say that a section is tight/coherent if the corresponding subdivision is. 
This is just an idea, and we are still working on it.

Finally, Ziegler claims that the point corresponding to this section in $\Sigma(P,Q)$ is a vertex because it maximizes $c$. To see this, notice that by construction, $c\cdot \gamma^c(x)\geq c \cdot \gamma(x)$ for any section $\gamma$. Then, 
\begin{displaymath}
\begin{split}
c\cdot \left( \frac{1}{\vol Q} \int_Q \gamma^c(x)dx\right) & =  \frac{1}{\vol Q} \int_Q c\cdot \gamma^c(x)dx \geq \frac{1}{\vol Q} \int_Q c\cdot \gamma(x)dx \geq  \\
& \geq c\cdot \left( \frac{1}{\vol Q} \int_Q \gamma(x)dx\right)
\end{split}
\end{displaymath}
So the point corresponding to $\gamma^c$ is the only point reaching equality in a valid inequality, and thus it is a vertex.

Now one need to see that from each vertex and its corresponding defining hyperplane, one can obtain a tight subdivision of $Q$. 
We have not completed this yet, and we still need to complete the last part of the proof. 









\end{document}